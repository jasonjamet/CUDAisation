\documentclass[serif,mathserif]{beamer}
\usepackage{graphicx}
\usepackage[utf8]{inputenc}
\usepackage[frenchb]{babel}

\usepackage{beamerthemesplit}
\definecolor{Silver1}{RGB}{0,93,117}
\definecolor{Silver2}{RGB}{50,50,50}
\definecolor{Silver3}{RGB}{0,142,178}


\setbeamercolor{frametitle}{use=structure,fg=white,bg=Silver1}
\setbeamercolor{structure}{fg=Silver3}
\setbeamercolor{section in head/foot}{bg=Silver2}
\setbeamercolor{author in head/foot}{bg=Silver2}
\setbeamercolor{date in head/foot}{fg=Silver2}



%%%%%%%%%%%%%%%%%%%%%%%UNIVERSITY LOGO DEFINITION%%%%%%%%%%%%%%%%%%%%%

\usepackage[absolute,overlay]{textpos}
  \setlength{\TPHorizModule}{1mm}
  \setlength{\TPVertModule}{1mm}
\newcommand{\logoUniv}{
	\begin{textblock}{0}(115,20)
      \includegraphics[scale=0.12,viewport=0 0 0 0]{angers.png}
    \end{textblock}	
}


%%%%%%%%%%%%%%%%%%%%%%%%%%%%PRESENTATION PAGE%%%%%%%%%%%%%%%%%%%%%%%%%


\title[Stage KaliConseil\hspace{2em}\insertframenumber/\inserttotalframenumber]{CUDAisation du code impératif}
\date{Vendredi 03 juin 2016} %leave out for today's date to be insterted

\institute[UFR Sciences]{Université d'Angers}


\begin{document}

\begin{frame}[plain]
\maketitle
\scriptsize

	\begin{minipage}{0.49\textwidth} 
		\begin{flushleft}
			\textbf{Tuteur pédagogique}:Jean Michel RICHER
    	\end{flushleft}
    \end{minipage}
    \begin{minipage}{0.49\textwidth} 
		\begin{flushright}
		Alexis BRIARD\\
		Jason JAMET\\
		Guillaume GRANDJEAN	
    		\end{flushright}
    \end{minipage}

\end{frame}
%%%%%%%%%%%%%%%%%%%%%%%%%%%%TABLE OF CONTENT%%%%%%%%%%%%%%%%%%%%%%%%%%


\frame{
	\frametitle{Plan} 
	\logoUniv \footnotesize
	\tableofcontents 
}
%%%%%%%%%%%%%%%%%%%%%%%%%%%%%%%%SLIDES%%%%%%%%%%%%%%%%%%%%%%%%%%%%%%%%

\section{Présentation} 
\subsection{Objectifs}
\frame{\frametitle{Objectifs} \logoUniv
\begin{itemize}
\item Esn créée en 2010
\item Basée sur Angers 
\item Conduite de projets
\item Infogérance (tpe et pme)
\end{itemize} 

}


\subsection{Technologies et outils utilisés}
\frame{\frametitle{La solution KS2} \logoUniv

}


\section{Analyse du problème}
\subsection{Analyse générale}
\frame{\frametitle{Analyse générale} \logoUniv

\begin{block}{Les transformations à réaliser :}
\begin{itemize}
\item Initialisation de la variable récupérée dans le thread\_loop
\item Remplacement de la boucle for en condition if
\item Modification de l'entête de la fonction
\end{itemize} 
\end{block}

}

\frame{\frametitle{Analyse générale} \logoUniv

\begin{block}{Eléments qui peuvent poser problème :}
\begin{itemize}
\item La variable size
\item La taille des blocs
\item Recherche de la boucle à paralléliser
\end{itemize} 
\end{block}

\begin{block}{Informations à extraire du code :}
\begin{itemize}
\item Pragma
\item Boucle for
\item Déclaration de variables
\item Fonction non void
\end{itemize} 
\end{block}

}






\subsection{Solutions envisagées}
\frame{\frametitle{Solutions envisagées}  \logoUniv
 
\begin{block}{Les solutions:}
\begin{itemize}
\item Analyseur syntaxique / lexical
\item Compilateur gcc
\item Regex
\end{itemize} 
\end{block}

}

\section{Conception}
\subsection{Solution retenue}
\frame{\frametitle{Solutions envisagées}  \logoUniv
\begin{block}{Analyseur syntaxique}
 
\end{block}
}
\subsection{Réalisation}
\frame{\frametitle{Base de travail et correction}  \logoUniv


}
\frame{\frametitle{Analyse du pragma}  \logoUniv


}

\frame{\frametitle{Recherche de la boucle à paralléliser} \logoUniv


}

\frame{\frametitle{Vérification de la portée des variables}\logoUniv


}

\frame{\frametitle{Transformation de la fonction}  \logoUniv


}

\frame{\frametitle{Wrapper de la fonction transformée}  \logoUniv


}

\section{Améliorations}
\subsection{Améliorations}
\frame{\frametitle{Améliorations}  \logoUniv
\begin{itemize}
\item Parse complet du langage c
\item Gestion plus complète de la boucle for
\item Automatisation des allocations
\end{itemize} 
}
 


\section{Conclusion}
\subsection{Conclusion}
\frame{\frametitle{Conclusion}  \logoUniv

}
\end{document}

