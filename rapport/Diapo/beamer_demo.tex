\documentclass[serif,mathserif]{beamer}
\usepackage{graphicx}
\usepackage[utf8]{inputenc}
\usepackage[frenchb]{babel}

\usepackage{beamerthemesplit}
\definecolor{Silver1}{RGB}{0,93,117}
\definecolor{Silver2}{RGB}{50,50,50}
\definecolor{Silver3}{RGB}{0,142,178}


\setbeamercolor{frametitle}{use=structure,fg=white,bg=Silver1}
\setbeamercolor{structure}{fg=Silver3}
\setbeamercolor{section in head/foot}{bg=Silver2}
\setbeamercolor{author in head/foot}{bg=Silver2}
\setbeamercolor{date in head/foot}{fg=Silver2}



%%%%%%%%%%%%%%%%%%%%%%%UNIVERSITY LOGO DEFINITION%%%%%%%%%%%%%%%%%%%%%

\usepackage[absolute,overlay]{textpos}
  \setlength{\TPHorizModule}{1mm}
  \setlength{\TPVertModule}{1mm}
\newcommand{\logoUniv}{
	\begin{textblock}{0}(115,20)
      \includegraphics[scale=0.12,viewport=0 0 0 0]{angers.png}
    \end{textblock}	
}


%%%%%%%%%%%%%%%%%%%%%%%%%%%%PRESENTATION PAGE%%%%%%%%%%%%%%%%%%%%%%%%%

\author[Jason JAMET]{Jason JAMET}

\title[Stage KaliConseil\hspace{2em}\insertframenumber/\inserttotalframenumber]{Développement de modules alfresco}
\date{Jeudi 25 juin 2015} %leave out for today's date to be insterted

\institute[UFR Sciences]{Université d'Angers}


\begin{document}

\begin{frame}[plain]
\maketitle
\scriptsize

	\begin{minipage}{0.49\textwidth} 
		\begin{flushleft}
			\textbf{Tuteur pédagogique}: Benoit DAMOTA
    	\end{flushleft}
    \end{minipage}
    \begin{minipage}{0.49\textwidth} 
		\begin{flushright}
			\textbf{Maitre de stage}: Vincent DEREC
    	\end{flushright}
    \end{minipage}

\end{frame}
%%%%%%%%%%%%%%%%%%%%%%%%%%%%TABLE OF CONTENT%%%%%%%%%%%%%%%%%%%%%%%%%%


\frame{
	\frametitle{Plan} 
	\logoUniv \footnotesize
	\tableofcontents 
}
%%%%%%%%%%%%%%%%%%%%%%%%%%%%%%%%SLIDES%%%%%%%%%%%%%%%%%%%%%%%%%%%%%%%%

\section{KaliConseil} 
\subsection{Présentation}
\frame{\frametitle{KaliConseil} \logoUniv
\begin{itemize}
\item Esn créée en 2010
\item Basée sur Angers 
\item Conduite de projets
\item Infogérance (tpe et pme)
\end{itemize} 
\hfill \includegraphics[scale=0.5,viewport=-30 50 320 170]{KaliConseil.jpg}
}


\subsection{KS2}
\frame{\frametitle{La solution KS2} \logoUniv
\begin{center}
	\includegraphics[scale=0.25]{KS2Product.png}
\end{center}
}


\section{Afresco}
\subsection{Présentation}
\frame{\frametitle{L'ECM Alfresco} \logoUniv
\begin{center}
	\includegraphics[scale=0.20]{alfrescoSchema.png}
\end{center}

}
%\subsection{Architecture}
%\frame{\frametitle{blocs}  \logoUniv

%\begin{block}{title of the bloc}
%bloc text
%\end{block}

%}
\subsection{Workflow}
\frame{\frametitle{Workflow}  \logoUniv
\begin{block}{Objectif: }
Automatisation des processus métier
\end{block}
\begin{center}
\includegraphics[scale=0.70]{BPMNExemple.png}
\end{center}


}
\subsection{Customisation de share}
\frame{\frametitle{Nouvelles pages personnalisés}  \logoUniv
 
\begin{block}{Les technologies:}
\begin{itemize}
\item Spring Surf
\item AIKAU (Dojo)
\item YUI
\end{itemize} 
\end{block}

}

\section{Mes Projets}
\subsection{Nature des projets}
\frame{\frametitle{Nature des projets}  \logoUniv
\begin{block}{De nouveaux Workflows:}
\begin{itemize}
\item Workflow Facture
\item Workflow demande de vacances
\end{itemize} 
\end{block}
\begin{block}{De nouvelles pages WEB:}
\begin{itemize}
\item Page management d'aspects et propriétés
\item Page management de Workflows
\end{itemize}
\end{block}
}
\subsection{Workflow Facture}
\frame{\frametitle{Workflow  "Facture"}  \logoUniv
\begin{block}{Objectif: }
Qualifier une facture semi-automatiquement, ou manuellement.
\end{block}
\begin{center}
\includegraphics[scale=0.35]{diagramFacture.png}
\end{center}
}
\frame{\frametitle{Workflow "Facture"}  \logoUniv

\begin{center}
\includegraphics[scale=0.35]{TacheFacture.png}
\end{center}
}
\subsection{Page management de Workflows}
\frame{\frametitle{Page management de Workflows}  \logoUniv
\begin{block}{Objectif: }
Accès simple aux Workflows et tâches liées à l'utilisateur courant
\end{block}
\begin{center}
\includegraphics[scale=0.25]{PageWF.png}
\end{center}
}
\section{Bilan}
\subsection{Difficultés rencontrées}
\frame{\frametitle{Problèmes rencontrés}  \logoUniv
\begin{itemize}
\item Documentation faible
\item Frameworks jeunes
\item Difficultés techniques
\end{itemize} 
}
\subsection{A venir}
\frame{\frametitle{Ce qu'il reste à faire}  \logoUniv
\begin{itemize}
\item De nombreux nouveaux Workflows
\item Lier une solution d'océrisation et  le Workflow "Facture"
\item Terminer la page "management de Workflows"
\end{itemize} 
}
\subsection{Conclusion}
\frame{\frametitle{Conclusion}  \logoUniv
\begin{itemize}
\item Très bonne expérience professionnelle
\item Confirmation de mon projet professionnel
\end{itemize} 
}
\end{document}

